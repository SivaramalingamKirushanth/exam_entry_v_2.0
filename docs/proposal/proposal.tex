\documentclass[12pt]{article}
\usepackage{graphicx}
\usepackage{geometry}
\usepackage{setspace}
\usepackage{hyperref}
\usepackage{parskip}

\geometry{a4paper, margin=1in}
\setstretch{1.5}

\title{Examination Entry System}
\author{Group 09}
\date{October 2024}

\begin{document}

\maketitle

\begin{center}
    \justifying
    A project proposal for partial fulfilment of the course unit IT3162 - Group Project for the degree of Information Technology
\end{center}

\section*{Group Members}
\begin{itemize}
    \item I.M.C. Jeewantha – 2020ICT18
    \item E.W.A.P. Egodawitharana – 2020ICT24
    \item A.I.F. Ilma – 2020ICT48
    \item C.H. Hettiarachchi – 2020ICT57
    \item M.I.F. Ilma – 2020ICT64
    \item A.R. Wijesuriya – 2020ICT101
    \item L.M. Zahran – 2020ICT119
\end{itemize}

\section*{Supervisor}
\textbf{Dr. S. Kirushanth}\\
Senior Lecturer,\\
Department of Physical Science,\\
Faculty of Applied Science,\\
University of Vavuniya

\newpage

\section*{Declaration}
\justifying
We hereby declare that the project proposal submitted for evaluation of course module IT3162 leading to the award of a Bachelor of Science in Information Technology is entirely our own work, and the contents taken from the work of others have been cited and acknowledged within the text. This proposal has not been submitted for any degree at this University or any other institution.\\

\noindent
\begin{tabbing}
    I.M.C. Jeewantha – 2020ICT18 \hspace{3cm} \= \underline{\hspace{4cm}} \\
    E.W.A.P. Egodawitharana – 2020ICT24 \> \underline{\hspace{4cm}} \\
    A.I.F. Ilma – 2020ICT48 \> \underline{\hspace{4cm}} \\
    C.H. Hettiarachchi – 2020ICT57 \> \underline{\hspace{4cm}} \\
    M.I.F. Ilma – 2020ICT64 \> \underline{\hspace{4cm}} \\
    A.R. Wijesuriya – 2020ICT101 \> \underline{\hspace{4cm}} \\
    L.M. Zahran – 2020ICT119 \> \underline{\hspace{4cm}}
\end{tabbing}
\vspace{2in}

I recommended to project to be carried out by the students.\\

\vspace{0.6in}

\begin{minipage}[t]{0.6\textwidth}
    ................................ \\ 
    Dr. S. Kirushanth \\ 
    Senior Lecturer, \\ 
    Department of Physical Science, \\ 
    Faculty of Applied Science, \\ 
    University of Vavuniya
\end{minipage}%
\begin{minipage}[t]{0.3\textwidth}
    \centering
     ................................ \\ 
    Date
\end{minipage}

\newpage

\tableofcontents

\newpage

\section{Introduction}
\justifying
\subsection{Introduction}
The "Examination Entry System" is a digital platform aimed at streamlining exam registration and verification processes at Universities/Faculties. The system allows students to apply for exams, management to verify eligibility based on attendance and disciplinary records, and administrators to manage student records. Additionally, the system enables staff to print attendance sheets based on applied exam entries, ensuring smooth exam management.

\subsection{Objectives}
\begin{itemize}
    \item Provide a platform for students to apply for exams.
    \item Enable management staff to verify eligibility based on predefined criteria such as attendance and discipline.
    \item Allow staff to print attendance sheets based on the verified exam entries.
    \item Empower heads of departments to monitor student entries, get the counts of each batch exam entries, and oversee verification.
    \item Give administrative full control to the faculty dean for record management and final adjustments.
    \item Print final admissions and attendance sheets for students and exam management.
\end{itemize}

\subsection{Benefits of this Research}
\begin{itemize}
    \item \textbf{Efficiency:} The system automates the application, verification, and attendance sheet generation process, reducing administrative workload.
    \item \textbf{Accuracy:} Automatically generating attendance sheets ensures they reflect accurate and up-to-date exam entries.
    \item \textbf{Transparency:} Real-time data for students, staff, and heads of departments ensures clarity in the exam entry process.
    \item \textbf{Security:} Role-based access controls and data encryption protect the integrity of student records and exam entries.
    \item \textbf{Accountability:} Comprehensive records enable easy tracking of each student’s application and attendance.
\end{itemize}

\newpage

\section{Background}
\justifying
\subsection{Background}
Traditional exam registration processes often involve manual paperwork, which can be prone to errors, inconsistencies, and delays. This creates challenges in managing large student populations and ensuring accurate exam entries. The proposed system aims to address these issues by automating key steps in the process, from application to attendance sheet generation.

\subsection{Review of the Existing Systems}
The current exam entry system is predominantly manual, involving extensive paperwork and human intervention. Students are required to submit paper-based exam applications, which must be manually processed by staff. The verification process, determining student eligibility based on attendance and disciplinary records, is also manual, leading to delays, errors, and inefficiencies.

\newpage

\section{Material and Methods}
\justifying
\subsection{Brief Description of Proposed System Design}
The system will support three main user roles:
\begin{itemize}
    \item \textbf{Students:} Apply for exams through a secure login interface. Applications are saved in the system, pending verification.
    \item \textbf{Management (Staff):} Verify student eligibility based on attendance and disciplinary records. Staff can also print attendance sheets based on the verified exam entries, simplifying the preparation process for upcoming exams.
    \item \textbf{Management (Heads of Departments):} View student records and exam entry counts for each batch, ensuring all eligible students are processed. Heads of departments oversee the verification process and provide final approval for exam entries.
    \item \textbf{Admin (Faculty Dean):} Has full authority to modify, view, and update records without the ability to delete them, ensuring data integrity. The admin oversees all student records, including any modifications, and ensures that final admissions are correctly printed.
\end{itemize}

\newpage

\section{Expected Results}
\justifying
\begin{itemize}
    \item A fully functional user-friendly system that allows for the easy application, verification, and attendance tracking of students' exam entries.
    \item Automatic generation of attendance sheets based on the applied and verified exam entries.
    \item Enhanced transparency and accuracy in handling exam registration and attendance records.
    \item Streamlined administrative control, reducing manual efforts while maintaining secure and reliable data management.
    \item Final admissions and attendance sheets ready for printing to ensure smooth examination logistics.
\end{itemize}

\newpage

\section{Timeline of the Research}
\justifying
\subsection{Gantt Chart and Table}
\begin{figure}[h]
    \centering
    \includegraphics[width=0.7\textwidth]{table.png} 
    \caption{Research Timeline Table}
    \label{fig:timeline_table}
\end{figure}

\begin{figure}[h]
    \centering
    \includegraphics[width=0.7\textwidth]{gantt.png} 
    \caption{Gantt Chart for Research Timeline}
    \label{fig:timeline_gantt}
\end{figure}

\newpage

\section{References}
\justifying
We received guidance from our supervisor regarding project design, requirements, and recommended software tools to develop the system.\\

In addition, various online resources were instrumental in ensuring accuracy and successful outcomes for this project:
\begin{itemize}
    \item \url{https://nextjs.org}
    \item \url{https://tailwindcss.com}
    \item \url{https://ui.shadcn.com}
\end{itemize}

Our Examination Entry System brings several improvements and unique features compared to existing systems. Some key enhancements include:

\begin{itemize}
    \item \textbf{Automated Eligibility Verification:} Unlike many systems where checks are done manually, ours streamlines the verification process, minimizing errors.
    \item \textbf{Role-Specific Access:} Our system gives HODs enhanced insights and admin privileges to ensure data integrity with limited deletion.
    \item \textbf{Attendance Sheet Printing:} Post-verification, staff can efficiently print attendance sheets, benefiting record-keeping and operations.
    \item \textbf{Streamlined Final Admission Printing:} Centralized printing ensures tangible records, reducing confusion.
    \item \textbf{Enhanced Security:} Role-specific permissions improve security, essential for institutional accountability.
\end{itemize}

The chosen tech stack includes:
\begin{itemize}
    \item \textbf{Frontend - Next.js:} Efficient for SEO and server-side rendering.
    \item \textbf{Backend - Node.js:} Handles multiple requests, ideal for real-time applications.
    \item \textbf{Database - MySQL:} Reliable for managing structured data, supporting complex queries and transactions.
\end{itemize}

MySQL’s popularity stems from its reliability, ease of use, and free availability. It efficiently manages large datasets and ensures data accuracy, making it a preferred choice for applications needing robust data management.

\end{document}
